\chapter{Einleitung}

Die Anzahl der pflegebedürftigen Menschen in Deutschland ist von 1999 bis 2019 stetig gestiegen. Waren es 1999 noch ca. 2 Millionen Pflegebedürftige, so hat sich die Zahl der Pflegebedürftigen Ende 2019 auf ca. 4,13 Millionen mehr als verdoppelt \cite[vgl.][]{PflegebedarfStatista}.

Für diese Menge an Pflegebedürftigen steht in Pflegeheimen im Jahre 2019 ein Personal von ca. 796.000 Pflegekräften zur Verfügung \cite[vgl.][]{PflegepersonalStatista}. Daraus ergibt sich eine Quote von ca. 5 Pflegebedürftigen pro Pflegekraft.

Ehrenamtliche Helfer:innen können demnach die Pflegeheime unterstützen und entlasten. Dabei werden teilweise sogenannte Aktivierungen durchgeführt. Diese sind ca. 45 bis 60 minütige Aktivitäten, die die ehrenamtlichen Helfer:innen oder die Pflegekräfte mit den Pflegebedürftigen unternehmen. Hierbei kann es sich z.B. um Spiele, Vorlesungen oder ähnliches handeln. Aktivierungen müssen häufig mehrere Stunden vorbereitet und geplant werden. Sollte eine Pflegekraft eine Aktivierungen vorbereiten, so geschieht dies in der Regel außerhalb der regulären Arbeitszeit.

Ziel dieser Studienarbeit ist es, eine Web-Anwendung bzw. Blog zu erstellen, auf der ehrenamtliche Helfer:innen und Pflegekräfte Wissen über Aktivierungen austauschen und bereits geplante Aktivierungen vorstellen können. So soll der Zeitaufwand für durchzuführende Aktivierungen verringert werden.

Um dieses Ziel zu erreichen werden im Laufe der Studienarbeit zwei Anwendungen erstellt. Die erste Anwendung ist die Web-Anwendung bzw. der Blog (Frontend), die die Informationen über die Aktivierungen (Blogbeiträge) anzeigt. Als zweites wird eine Anwendung zum Verwalten der Blogbeiträge (Backend) entwickelt, dass u.a. das Erstellen von neuen Beiträgen vereinfachen soll.

Es werden Grundkenntnisse von HTML, CSS, JavaScript, Node.js bzw. npm, Docker bzw. docker-compose und Bash vorausgesetzt.
